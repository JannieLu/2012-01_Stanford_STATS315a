% ****** Start of file apssamp.tex ******
%
%   This file is part of the APS files in the REVTeX 4 distribution.
%   Version 4.0 of REVTeX, August 2001
%
%   Copyright (c) 2001 The American Physical Society.
%
%   See the REVTeX 4 README file for restrictions and more information.
%
% TeX'ing this file requires that you have AMS-LaTeX 2.0 installed
% as well as the rest of the prerequisites for REVTeX 4.0
%
% See the REVTeX 4 README file
% It also requires running BibTeX. The commands are as follows:
%
%  1)  latex apssamp.tex
%  2)  bibtex apssampya
%  3)  latex apssamp.tex
%  4)  latex apssamp.tex
%
\documentclass[pra,groupedaddress,amsmath,amssymb, column]{revtex4}

\setlength{\parindent}{10mm}
%\documentclass[twocolumn,pra,groupedaddress,amsmath,amssymb]{revtex4}
%\documentclass[twocolumn,pra,showpacs,groupedaddress,superscriptaddress,amsmath,amssymb]{revtex4}
%\documentclass[preprint,showpacs,preprintnumbers,amsmath,amssymb]{revtex4}

% Some other (several out of many) possibilities
%\documentclass[preprint,aps]{revtex4}
%\documentclass[preprint,aps,draft]{revtex4}
%\documentclass[pra]{revtex4}% Physical Review B

\usepackage{graphicx}% Include figure files
\usepackage{braket}
\usepackage{amsmath}
\usepackage{bm}% bold math
\usepackage{subfigure} 
%\usepackage{dcolumn}% Align table columns on decimal point


\begin{document}

\title{Stats 315a: Statistical Learning \\ Problem Set 1}
\author{Dong-Bang Tsai}
    \email{dbtsai@stanford.edu}
\affiliation{Department of Applied Physics, Stanford University, Stanford, California 94305, USA}



\date{\today}
\maketitle


\section*{Problem 1}
\subsection*{(a)}


\section*{Problem 2 (ESL 2.4)}
The squared distance from any sample point to the origin has a $\chi_p^2$ distribution with mean $p$; therefore, since a prediction point $x_0$ is drawn from this distribution, it will have a expected squared distance $p$ from the origin. 

Because $z_i=a^Tx_i$, and $a$ is independent from $x_i$, we can conclude that $z_i \sim N$, normal distribution. Now, let's calculate the expectation value of $z_i$. We know that for a $p$ dimensional vector $a$ 
\begin{align}
E(z) = E(a^Tx) = a^TE(x) = 0
\end{align}

The co-variant can be given by
\begin{align}
Cov(a^Tx) = a^Ta = 1
\end{align}
As a result, we get $z_i \sim N(0,1)$, and the expected squared distance will be $E(z^2) = 1$.



\section*{Problem 3}
\subsection*{(a)}



\subsection*{(b)}

\subsection*{(c)}

\subsection*{(d)}



\section*{Problem 4}
\subsection*{(a)}

\section*{Problem 5}
\subsection*{(a)}

\section*{Problem 6}
\subsection*{(a)}

\subsection*{(b)}
In the ridge regression approach, the equation we are trying to solve is 
\begin{align}
(X^TX +  \lambda I)\beta = X^T y
\end{align}
The solution can be given by $\hat{\beta}_{\lambda}=(X^TX +  \lambda I)^{-1}X^T y$.  We add positive constant to diagonal of $X^TX$; therefore, $(X^TX +  \lambda I)^{-1}$ is non-singular, even if $X^TX$ is not of full rank. As a result, the solution always exists, and is unique.
\subsection*{(c)}

\subsection*{(d)}
Using $X=UDV^T$
\begin{align}
\hat{\beta}_{\lambda}&=(X^TX +  \lambda I)^{-1}X^T y \nonumber\\
				       &=\left( VD^TU^TUDV^T  + \lambda I \right)VDU^T y \nonumber\\
                                      &=V\left(D^2 + \lambda I \right)V^TDU^T y \nonumber\\
                                      &=V\left(D^2 + \lambda I \right)DU^T y
\end{align}
\end{document}

